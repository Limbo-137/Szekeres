\chapter{拓扑}
到目前为止,我们主要讨论了代数结构在数学物理中的作用。虽然前一章提到了微积分,但并没有系统地展开。像\emph{连续性}和\emph{可微性}这样的\emph{分析学}核心概念,其本质是几何的,需要依赖\emph{拓扑}来进行严格定义。从广义的角度,\emph{拓扑}是加在几何上的结构,用以定义序列或子集的\emph{收敛}和\emph{极限}。定义了拓扑的空间叫做\emph{拓扑空间},拓扑空间之间的\emph{连续映射}是保持子集极限点的映射。研究拓扑学最一般的方法是从开集的概念入手。

想象一个嵌入到三维欧几里得空间 $\mathbb{E}^{3}$ 中的二维表面 $S$ 。这种情形下我们可以直观地将“连续变形”理解为曲面的一种不会发生撕裂或粘贴的变形过程。拓扑学主要研究的是那些在连续变形下不变的属性。度量性质并非连续性的关键,像“拉伸”这样的操作是可以允许的,因此拓扑学有时被称为“橡皮片几何”。在本章我们还将定义\emph{度量空间}的概念。这一类空间可以自然地定义出拓扑,但反过来并不一定成立——很多带有拓扑的空间并没有度量的概念。

\section{欧氏拓扑}

实数轴和欧几里得平面 $\mathbb{R}^{2}$ 是拓扑空间的原型。在实数轴 $\mathbb{R}$ 上,一个开区间是形如 $(a,b)=\{x\in \mathbb{R} \mid a< x< b\}$ 的集合。若存在 $\epsilon  >0$,使得开区间 $(x-\epsilon ,x+\epsilon )$ 完全包含在集合 $U\subseteq \mathbb{R}$ 中,则集合$U$称为点$\ x\in \mathbb{R} \ $的邻域。给定一个实数序列$\{x_{n} \}$,若对于任意的 $\epsilon  >0$,都存在一个整数 $N >0$,使得对于所有 $n >N$,都有 $|x_{n} -x|< \epsilon $,则称 $\{x_{n} \}$ 收敛于 $x\in \mathbb{R}$,记作 $x_{n}\rightarrow x$。也就是说,随着 $ n$ 趋近于无穷大,序列 $x_{n}$ 会进入并停留在每个包含 $x$ 的邻域 $U$ 内。此时,点 $x$ 被称为序列 $\{x_{n} \}$ 的极限。

\begin{exercise}
    证明序列的极限是唯一的:如果 $x_{n}\rightarrow x$ 且 $x_{n}\rightarrow x'$,那么 $x=x'$。
\end{exercise}

    类似的定义适用于欧几里得平面 $\mathbb{R}^{2}$,我们设定$|y-x|=\sqrt{(y_{1} -x_{1} )^{2} +(y_{2} -x_{2} )^{2}}$。在这种情况下,开区间的概念被开球取代:
\begin{equation*}
B_{r} (x)=\{y\in \mathbb{R}^{2} \mid |y-x|< r\}
\end{equation*}
如果存在实数 $\epsilon  >0$,使得开球 $B_{\epsilon } (x)\subset U$,则称集合 $U\subset \mathbb{R}^{2}$ 是点 $x\in \mathbb{R}^{2}$ 的一个邻域。对于一个点列 $\{x_{n} \}$ ,若对于每一个 $\epsilon  >0$,都存在一个正整数 $N >0$,使得对于所有 $n >N$,都有
\begin{equation*}
x_{n} \in B_{\epsilon } (x)。
\end{equation*}
则称点列 $\{x_{n} \}$ 收敛于 $x\in \mathbb{R}^{2}$,或者说 $x$ 是点列 $\{x_{n} \}$ 的极限,写作 $x_{n}\rightarrow x$,这个定义等价于这样的表述:对于 $x$ 的每个邻域 $U$,都存在一个 $N >0$,使得对于所有 $n >N$,都有 $x_{n} \in U$。
    在 $\mathbb{R}$ 或 $\mathbb{R}^{2}$ 中,开集 $U$ 是指包含其每个点的邻域的集合。直观地说,$U$ 在 $\mathbb{R}$(或 $\mathbb{R}^{2}$)中是开集,当且仅当 $U$ 中的每个点都可以被“扩展”成一个开区间(或开球)(见图 10.1)。例如,单位球 $B_{1} (O)=\{y\mid |y|^{2} < 1\}$ 是一个开集,因为对于每个点 $x\in B_{1} (O)$,都有开球 $B_{\epsilon } (x)\subset B_{1} (O)$,其中 $\epsilon =1-|x| >0$。
    在实数线上,可以证明最一般的开集是由不相交的开区间组成的,可以写成:
\begin{equation*}
\dotsc ,(a_{-1} ,a_{0} ),(a_{1} ,a_{2} ),(a_{3} ,a_{4} ),(a_{5} ,a_{6} ),\dotsc 
\end{equation*}
其中 $\dotsc a_{-1} < a_{0} \leq a_{1} < a_{2} \leq a_{3} < a_{4} \leq a_{5} < a_{6} \leq \dotsc $。在 $\mathbb{R}^{2}$ 中,开集不能如此简单地分类,因为尽管每个开集都是开球的并集,但这些并集不一定是互不相交的。
在标准分析中,函数 $f:\mathbb{R}\rightarrow \mathbb{R}$ 在点 $x$ 连续当且仅当对于任意的 $\epsilon  >0$,都存在 $\delta  >0$,使得
\begin{equation*}
|y-x|< \delta \ \ \Rightarrow \ \ |f(y)-f(x)|< \epsilon 。
\end{equation*}
因此,对于任意的 $\epsilon  >0$,逆像集 $f^{-1} (f(x)-\epsilon ,f(x)+\epsilon )$ 是 $x$ 的一个邻域,因为它包含一个以 $x$ 为中心的开区间 $(x-\delta ,x+\delta )$。由于每个 $f(x)$ 的邻域都包含一个形如 $(f(x)-\epsilon ,f(x)+\epsilon )$ 的区间,因此,函数 $f$ 在 $x$ 连续,当且仅当每个 $f(x)$ 邻域的逆像也是 $ x$ 的邻域。若一个函数 $ f:\mathbb{R}\rightarrow \mathbb{R}$ 在每点 $ x\in \mathbb{R}$ 连续,则称 $ f$ 在 $ \mathbb{R}$ 上连续。

\begin{theorem}\label{thm:10.1} 
     函数 $f:\mathbb{R}\rightarrow \mathbb{R}$ 在 $\mathbb{R}$ 中连续当且仅当对于每个开集 $V\subset \mathbb{R}$,其逆像 $f^{-1} (V)$ 是 $\mathbb{R}$ 的一个开集。
\end{theorem}
\begin{proof}
设函数 $ f$ 在 $ \mathbb{R}$ 上连续,因为开集对于每一点 $ y\in U$ 都是邻域,则 $ U$ 的逆像 $ V=f^{-1}( U)$ 必须是其中每点 $ x\in V$ 的邻域。因此 $ V$ 是一个开集。
    反过来,设函数 $f:\mathbb{R}\rightarrow \mathbb{R}$ 满足对于每个开集 $U\subset \mathbb{R}$,其逆像 $f^{-1} (U)$ 是开集。对任意 $x\in \mathbb{R}$ 为和任意 $\epsilon  >0$ ,$(f(x)-\epsilon ,f(x)+\epsilon )$ 的逆像是一个包含 $ x$ 的开集。它因此包含了一个形如 $ ( x-\delta ,\ x+\delta )$ ,于是 $ f$ 在点 $ x$ 是连续的。又因为 $ x$ 点是在 $\mathbb{R}$ 上任取的,因此 $ f$ 在 $\mathbb{R}$ 上连续。
\end{proof}
    在更广义的拓扑学上,上述关系将被用于定义连续映射。$\mathbb{R}^{2}$ 中对于连续的处理方法和 $ \mathbb{R}$ 几乎相同。函数 $f:\mathbb{R}^{2}\rightarrow \mathbb{R}^{2}$ 在点 $x$ 连续当且仅当对于每个 $\epsilon  >0$,都存在 $\delta  >0$,使得
\begin{equation*}
|y-x|< \delta \ \ \Rightarrow \ \ |f(y)-f(x)|< \epsilon 。
\end{equation*}
利用和定理\ref{thm:10.1}几乎相同的证明方法,可得知函数 $f$ 在 $ \mathbb{R}$ 上连续当且仅当对于每个开集 $U\subset \mathbb{R}^{2}$,其逆像 $f^{-1} (U)$ 是 $\mathbb{R}^{2}$ 的开子集。对实值函数 $f:\mathbb{R}^{2}\rightarrow \mathbb{R}$ 也是如此。于是函数的连续性可以完全通过它们对陪域上开集的逆作用来描述。正因如此,开集被视为拓扑空间的关键成分。处理欧几里得空间及其内嵌曲面的经验使得数学家们认识到,开集的最重要特性可以用几个简单的规则来总结,这些规则将在下一节中列出(另见 [1–8])。

\section{广义拓扑空间}
(填充)
\section{矩阵空间}
(填充)
\section{诱导拓扑}
(填充)
\subsection{诱导拓扑和拓扑乘积}
(填充)
\subsection{同化拓扑}
(填充)
\section{豪斯多夫空间}

在某些拓扑中(例如离散拓扑)由于开集太少,不同的点无法被非相交的邻域分隔开。为了解决这种情况,有时会对拓扑空间施加称为\emph{分离公理}的条件。其中一种最常见的是\textbf{豪斯多夫条件(Hausdorff condition)}:对于每一对点$x,y\in X$,存在$x$的开邻域 $U$ 和 $y$ 的开邻域 $V$ ,使得$U\cap V=\emptyset $。满足此属性的拓扑空间称为\textbf{豪斯多夫空间(Hausdorff space)}。直观来讲,豪斯多夫空间中不存在一对“任意接近”的不同点。

豪斯多夫空间的一个典型的“好的”属性是,任何收敛序列 $x_{n}\rightarrow x$ 的极限(定义见\ref{pro:10.7})是唯一的。举例来讲,若在豪斯多夫空间$X$中,$x_{n}\rightarrow x$且$x_{n}\rightarrow x'$。假设$x\neq x'$,给出不相交的开邻域$U$和$U'$,使得$x\in U$且$x'\in U'$,并且存在一个整数$N$使得对所有$n >N$,$x_{n} \in U$。在此情况下,由于对所有$n >N$,$x_{n} \notin U'$,因此序列$x_{n}$不能收敛于$x'$,与条件$x_{n}\rightarrow x'$矛盾。于是只能有$x=x'$。

在豪斯多夫空间中,每个单点集$\{x\}$都是闭集。原因如下,令$Y=X-\{x\}$为其补集。$Y$ 中的每个点$y$都有一个不与$x$的某个开邻域相交的开邻域$U_{y}$,特别是对于$x\notin U_{y}$。根据\ref{top:3},所有这些开邻域的并集$Y=\bigcup _{y\in Y} U_{y} =X-\{x\}$是开的。因此 $\{x\}=X-Y$ 为开集的补集是闭的。

\begin{exercise}
	证明在一个有限集$X$上,唯一的豪斯多夫拓扑是离散拓扑。正是由于这个原因,有限集的拓扑才比较无趣。
\end{exercise}

\begin{theorem}\label{eg:10.4}
	度量空间$(X,d)$都是是豪斯多夫空间。
\end{theorem}

\begin{proof}
	找任意不同点$x,y\in X$,并设 $\epsilon =\frac{1}{4} d(x,y)$ 。开球$U=B_{\epsilon } (x)$和$V=B_{\epsilon } (y)$分别是$x$和$y$的开邻域。它们的交集是空的,因为如果$z\in U\cap V$,则$d(x,z)< \epsilon $且$d(y,z)< \epsilon $,这与三角不等式\ref{met:4}
\begin{equation*}
d(x,y)\leq d(x,z)+d(z,y)\leq 2\epsilon < \frac{1}{2} d(x,y)
\end{equation*}
相矛盾。
\end{proof}

这个定理的一个直接推论是,对于所有$n >0$,标准拓扑在$\mathbb{R}^{n}$上是豪斯多夫的。

\begin{theorem}\label{eg:10.5}
	给定拓扑空间 $X$,$Y$ 与连续单射 $f:X\rightarrow Y$ 。若 $Y$ 是豪斯多夫的,则$X$是豪斯多夫的。
\end{theorem}

\begin{proof}
	设$x$和$x'$是$X$中的任意一对不同点,并设$y=f(x)$和$y'=f(x')$。由于$f$是单射,这些点在 $Y$ 同样是不同点。如果 $Y$ 是豪斯多夫的,则存在$y$和$y'$的不相交开邻域$U_{y}$和$U_{y} '$。这些集合相对 $f$ 的逆像分别是是 $x$ 和 $x'$ 的不相交开邻域,因为$f^{-1} (U_{y} )\cap f^{-1} (U_{y} ')=f^{-1} (U_{y} \cap U_{y} ')=f^{-1} (\emptyset )=\emptyset $。
\end{proof}

	这意味着豪斯多夫条件是一个真正的拓扑性质,其在拓扑变换下是不变的,因为若$f:X\rightarrow Y$是一个同胚映射,则$f^{-1} :Y\rightarrow X$是连续单射。

\begin{corollary}\label{eg:10.6}
	任何豪斯多夫空间的子空间在赋予相对拓扑后是豪斯多夫的。
\end{corollary}

\begin{proof}
	设$A$是拓扑空间$X$的任一子集。赋予其相对拓扑后,包含映射$i_{A} :A\rightarrow X$是连续的。由于它是一对一的,由定理\ref{thm:10.5}可知$A$是豪斯多夫空间。
\end{proof}

\begin{theorem}\label{eg:10.7}
	若$X$和$Y$是豪斯多夫拓扑空间,则它们的拓扑乘积$X\times Y$是豪斯多夫的。
\end{theorem}

\begin{proof}
	设$(x,y)$和$(x',y')$是 $X\times Y$ 中的任意两对不同点,即$x\neq x'$或$y\neq y'$。假设$x\neq x'$。那么存在$X$中的开集$U$,$U'$使得$x\in U$,$x'\in U'$,且$U\cap U'=\emptyset $。于是可以找到集合$U\times Y$和$U'\times Y$, 使得其是$(x,y)$和$(x',y')$的不相交开邻域。类似的,若$y\neq y'$,可以找到不相交邻域$X\times V$和$X\times V'$用来分隔这两个点。
\end{proof}

\section{紧空间}
(填充)
\section{连通空间}
(填充)
\section{拓扑群}

拓扑结构和代数结构可以用很多种有意义的方式结合起来。结合这两种结构的核心要求是,用于表示代数运算法则的函数必须是拓扑连续的(相对于底拓扑空间)。在这一节中,我们将群论与拓扑学结合起来。
	一个\textbf{拓扑群(topological group)}是指一个既是群又是豪斯多夫空间的集合 $ G$ ,其中由 $\psi :G\times G\rightarrow G$ 定义的映射 $\psi (g,h)=gh^{-1}$ 是连续的。如果底空间的拓扑是离散的,则称拓扑群 $G$ 为\textbf{离散的(discrete)}。
	由 $\varphi (g,h)=gh$ 和 $\psi (g,h)=gh^{-1}$ 定义的映射 $\varphi :G\rightarrow G$ 和 $\psi :G\times G\rightarrow G$ 都是连续的。因为根据定理\ref{thm:10.1},包含映射 $i:G\rightarrow G\times G$ 定义为 $i(h)=\iota _{e} (e,h)$ ,其为连续映射。于是映射 $\tau =\psi \circ i$ 也是连续的,因为它是连续映射$ \psi $和$ i$的复合映射。而 $\phi ( g,h) =\psi (g,\tau (h))$,可知 $\phi $ 也是连续映射。

\begin{exercise}
	证明 $\varphi $ 是 $G$ 的同胚映射。
\end{exercise}

\begin{exercise}
	如果 $\varphi $ 和 $\psi $ 是连续映射,证明 $\varphi $ 是连续的。
\end{exercise}

\begin{eg}\label{eg:10.20}
加法群 $\mathbb{R}^{n}$,其群乘法是向量加法,
\begin{equation*}
\phi (\mathbf{x} ,\mathbf{y}) =(x^{1} ,x^{2} ,\dotsc ,x^{n} )+(y^{1} ,y^{2} ,\dotsc ,y^{n} )=(x^{1} +y^{1} ,\dotsc ,x^{n} +y^{n} )
\end{equation*}
逆映射为
\begin{equation*}
\tau (\mathbf{x} )=-\mathbf{x} =(-x^{1} ,\dotsc ,-x^{n} ),
\end{equation*}
这个群是一个建立在欧几里得拓扑上的阿贝尔拓扑群。$n$-环面 $T^{n} =\mathbb{R}^{n} /\mathbb{Z}^{n}$ 同样也是一个阿贝尔拓扑群,其群乘法是模 $ 1$ 加法。
\end{eg}

\begin{eg}\label{eg:10.20}
全体 $n\times n$ 实矩阵 $M_{n} (\mathbb{R} )$ 构成的集合具有与欧几里得拓扑 $\mathbb{R}^{n^{2}}$ 同胚的拓扑。行列式映射 $\det :M_{n} (\mathbb{R} )\rightarrow \mathbb{R}$ 显然是连续的,因为行列式 $ \ \mathrm{det}\mathbf{A}$ 是 $ \mathbf{A}$ 的矩阵元的多项式函数。于是一般线性群 $GL(n,\mathbb{R} )$ 是 $M_{n} (\mathbb{R} )$ 的一个开子集,因为其是开集 $\dot{\mathbb{R}} =\mathbb{R} -\{0\}$在行列式逆映射下的像。如果给 $GL(n,\mathbb{R} )$ 赋予由 $M_{n} (\mathbb{R} )$ 中的拓扑诱导出的相对拓扑,那么映射 $\psi $ 以分量形式写为
\begin{equation*}
( \psi (\mathbf{A} ,\mathbf{B} ))_{ij} =\sum _{k=1}^{n} A_{ik}\left( B^{-1}\right)_{kj} 。
\end{equation*}
这些分量的映射都是连续函数,因为 $(B^{-1} )_{ij}$ 是矩阵元 $B_{ij}$ 的有理多项式函数,且函数的分母为非奇异矩阵的多项式 $ \mathrm{det} B$ 。
\end{eg}

	群 $G$ 的一个子群 $H$ 连带着其相对拓扑一起被称为 $G$ 的\textbf{拓扑子群(topological subgroup)}。为了证明任何子群 $H$ 在相对拓扑下成为拓扑子群,设 $U'=H\cap U$,其中 $U$ 是 $G$ 的任意开子集。由于映射 $\phi $ 的连续性,对于任何一对使得 $gh\in U'\subset U$ 的点 $g,h\in H$,存在 $G$ 中的开集 $A$ 和 $B$,使得 $A\times B\subset \varphi ^{-1} (U)$。由此得出 $\varphi (A'\times B')\subset H\cap U$,其中 $A'=A\cap H$,$B'=B\cap H$,而 $ \phi |_{H}$ 的连续性是显然的。类似地,限制到 $H$ 上逆映射 $\tau $ 也是连续的。如果 $H$ 还是 $G$ 中的闭集,则称其为 $G$ 的\textbf{闭子群}。
	对于每个 $g\in G$,令左平移映射 $L_{g} :G\rightarrow G$ 为
\begin{equation*}
L_{g} (h)\equiv L_{g} h=gh
\end{equation*}
与例\ref{eg:2.25}中的定义相同。由于映射 $L_{g}$ 是两个连续映射的复合映射,$L_{g} =\phi \circ \iota _{g}$,因此 $L_{g}$ 是连续的,其中 $\iota _{g} :G\rightarrow G\times G$ 是包含映射 $\iota _{g} (h)=(g,h)$。
(见定理\ref{thm:10.3})。它显然是单射,因为 $gh=gh'\Longrightarrow h=g^{-1} gh'=h'$ ,并且它的逆是连续映射 $L_{g^{-1}}$。由此可知 $L_{g}$ 是同胚映射。类似地,对于每个右平移可以定义 $R_{g} :G\rightarrow G$ 为 $R_{g} h=hg$,其也是 $G$ 的同胚映射,还可以定义内自同构 $C_{g} :G\rightarrow G$ 为 $C_{g} h=ghg^{-1} =L_{h} \circ R_{h^{-1}}( g)$ 。

\subsection{单位元的连通分支}

如果 $G$ 是一个拓扑群,我们将包含单位元的连通分支记为 $ G_{0}$ ,简称为\textbf{单位连通分支(component of the identity)}。

\begin{theorem}\label{eg:10.20}
	设 $G$ 是一个拓扑群,$G_{0}$ 是其单位连通分支。则 $G_{0}$ 为 $G$ 的闭正规子群。
\end{theorem}

\begin{proof}
	根据定理\ref{thm:10.17},集合 $G_{0} g^{-1}$ 是连通的,因为它是右平移 $g^{-1}$ 对连通集的连续映射。若 $g\in G_{0}$,则 $e=gg^{-1} \in G_{0} g^{-1}$。因此,$G_{0}$ 是一个包含单位元的闭连通子集,且其为 $G_{0}$ 的子集。所以我们得到 $G_{0} G_{0}^{-1} \subseteq G_{0}$,从而可知 $G_{0}$ 是 $G$ 的子群。又由于 $G_{0}$ 是 $G$ 的连通分支,则它是 $G$ 的闭子群。
	对于任意的 $g\in G$,集合 $gG_{0} g^{-1}$ 是连通的,因为它是 $G_{0}$ 在内自同构映射 $h\rightarrow C_{g} (h)$ 下的像。由于该集合包含单位元 $e$,我们得到 $gG_{0} g^{-1} \subseteq G_{0}$ ,于是 $G_{0}$ 是一个正规子群。
\end{proof}

一个拓扑空间 $X$ 中每个点的每个邻域都包含一个连通开邻域,则其称为\textbf{局部连通(locally connected)}的。若一个拓扑群 $G$ 在单位元 $e$ 处局部连通,则整个 $G$ 局部连通,因为如果 $V$ 是单位元 $e$ 的开连通邻域,那么选中任一点 $g\in G$ , $gV =L_{g} V$ 是开连通邻域。如果 $K$ 是群 $G$ 的任一子集,我们称包含 $K$ 的最小子群为\textbf{由 $K$ 生成的子群(subgroup generated by $ K$)}。它是包含 $K$ 的所有子群的交集。
一个拓扑空间 $X$ 中每个点的每个邻域都包含一个连通开邻域,则其称为\textbf{局部连通(locally connected)}的。若一个拓扑群 $G$ 在单位元 $e$ 处局部连通,则整个 $G$ 局部连通,因为如果 $V$ 是单位元 $e$ 的开连通邻域,那么选中任一点 $g\in G$ , $gV =L_{g} V$ 是开连通邻域。如果 $K$ 是群 $G$ 的任一子集,我们称包含 $K$ 的最小子群为\textbf{由 $K$ 生成的子群(subgroup generated by $ K$)}。它是包含 $K$ 的所有子群的交集。

\begin{theorem}\label{eg:10.21}
	在任何局部连通群 $G$ 中,单位连通分支 $G_{0}$ 可由单位元 $e$ 的任何连通邻域生成。
\end{theorem}

\begin{proof}
	设 $V$ 为单位元 $e$ 的任意连通邻域,$H$ 为由 $V$ 生成的子群。对于任意 $g\in H$, $L_{g}$ 是同胚映射,于是左陪集 $gV=L_{g} V\subset H$ 是 $g$ 的一个邻域。因此 $H$ 是 $G$ 的一个开子群。另一方面,由于 $H$ 是 $G$ 中 $H$ 陪集之并的补集,则 $H$ 同时也是闭集。由此,$H$ 既是开集又是闭集,其本身就是单位连通分支 $G_{0}$。
\end{proof}

	设 $H$ 为拓扑群 $G$ 的闭子群,我们可以通过典范投影映射 $\pi :g\mapsto gH$ 给商空间 $G/H$ 诱导出自然拓扑,该拓扑由标准投影映射 $\pi :g\mapsto gH$ 定义。这是 $G/H$ 上使得 $\pi $ 是连续映射的最大拓扑。在这个拓扑中,对于一组 $H$ 的陪集,若它们的并集是 $G$ 的一个开子集,则以每个陪集为元素其全体构成一个集合 $U\subset G/H$ ,其在 $G/H$ 中是开集。显然,在这个拓扑下$\pi $ 是一个\textbf{开映射(open map)},即是说所有开集 $V\subseteq G$ 的像 $\pi (V)$ 是开集。

\begin{theorem}\label{eg:10.21}
	如果 $G$ 是一个拓扑群,而 $H$ 是一个使商空间 $G/H$ 连通的闭连通子群,则 $G$ 是连通的。
\end{theorem}

\begin{proof}
	假设 $G$ 不是连通的。那么存在开集 $U$ 和 $V$,使得 $G=U\cup V$,且 $U\cap V=\emptyset $。$\pi $ 是开映射,于是映射 $\pi (U)$ 和 $\pi (V)$ 在 $G/H$ 中是开集,且 $G/H\in \pi (U)\cup \pi (V)$ 。但是 $G/H$ 是连通的,意味着 $\pi (U)\cap \pi (V)\neq \emptyset $ ,交界处存在陪集 $gH\in \pi (U)\cap \pi (V)$。而作为 $G$ 的一个子集,这个陪集显然与 $ U$ 和 $ V$ 都相交, $gH\in (gH\cap U)\cup (gH\cap V)$,与 $gH$ 连通的事实相矛盾(因为它是连通集 $ H$ 对于连续映射 $L_{g}$ 的像)。因此$G$ 是连通的。
\end{proof}

\begin{eg}\label{eg:10.22}
	一般线性群 $GL(n,\mathbb{R} )$ 不是连通的,因为行列式映射 $\det :GL(n,\mathbb{R} )\rightarrow \mathbb{R}$ 的像为 $\dot{\mathbb{R}} =\mathbb{R} -\{0\}$,这是一个不连通的集合。单位元 $ \mathbf{I}$ 的单位连通分支 $G_{0}$ 是行列式 $  >0$ 的 $n\times n$ 矩阵集合,以各分支作为元素构成的群是离散群
\begin{equation*}
GL(n,\mathbb{R} )/G_{0} \cong \{1,-1\}=Z_{2}
\end{equation*}
然而值得一提的的是,复一般线性群 $GL(n,\mathbb{C} )$ 是连通的,这一结论成立的依据是,任何非奇异复矩阵的 Jordan 标准型可以连续变形为单位矩阵 $ \mathbf{I}$ 。
\end{eg}

特殊正交群 $SO(n)$ 全都是连通的。可以通过利用维数 $n$ 进行数学归纳来证明这一点。显然 $SO(1)=\{1\}$ 是连通的。假设 $SO(n)$ 是连通的,则可以证明(参见\ref{chapter:19}的\ref{eg:19.10})$SO(n+1)/SO(n)$ 同胚于球面 $S^{n}$。$S^{n}$是一个连通集(\ref{eg:10.19}),由\ref{thm:10.22}可知 $ SO(n+1)$ 是连通集。于是利用数学归纳法可知,对于所有 $n=1,2,\dotsc $, $ SO(n)$ 是连通群。然而,正交群 $O(n)$ 并不是连通的,因为其单位连通分支是 $SO(n)$,剩余的正交矩阵都在另一个对应行列式为 $ -1$ 的连通分支上。

类似地,由 $SU(1)=\{1\}$ 和 $SU(n)/SU(n)\cong S^{2n-1}$,可知所有的特殊酉群 $SU(n)$ 是连通的。由\ref{thm:10.22},酉群 $U(n)$ 也都是连通的,因为 $U(n)/SU(n)=S^{1}$ 是连通的。

\section{拓扑向量空间}
(填充)
\subsection{巴拿赫空间}
(填充)